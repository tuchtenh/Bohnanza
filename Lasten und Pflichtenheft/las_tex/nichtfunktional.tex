\chapter{Nicht-funktionale Anforderungen}

\newcounter{nf}\setcounter{nf}{10}

\section{Softwarearchitektur}

\begin{description}[leftmargin=5em, style=sameline]	
	\begin{lhp}{nf}{NF}{nfunk:sarch1}
		\item [Name:] Client-Server Anwendung
		\item [Beschreibung:] Das verteilte Spiele-System ermöglicht das gemeinsame Spielen von verschiedenen Rechnern aus.
		\item [Motivation:] Aufgabestellung v. SEP/MP.
		\item [Erfüllungskriterium:] Das fertige System besteht aus Client- und Server-Teilen.
	\end{lhp}
	
	\begin{lhp}{nf}{NF}{nfunk:sarch1}
		\item [Name:] Plattformunabhängigkeit
		\item [Beschreibung:] Es soll sich um eine plattformunabhängige Anwendung handeln. Zumindest Windows- und Linuxsysteme sind zu unterstützen.
		\item [Motivation:] Aufgabenstellung v. SEP/MP.
		\item [Erfüllungskriterium:]In den Präsentationen kann der aktuelle Stand bzw. das fertige Produkt jederzeit vorgeführt werden.
	\end{lhp}
\end{description}



\section{Benutzerfreundlichkeit}


\begin{description}[leftmargin=5em, style=sameline]	
	\begin{lhp}{nf}{NF}{nfunk:alter}
		\item [Name:] Benutzeralter
		\item [Beschreibung:] Das System ist für Benutzer geeignet, die älter als 5 Jahre sind.
		\item [Motivation:] Jüngere Benutzer sind unfähig das Spiel zu spielen.
		\item [Erfüllungskriterium:] In AGB steht ein entsprechendes Hinweis.
	\end{lhp}
\end{description}

\begin{description}[leftmargin=5em, style=sameline]	
	\begin{lhp}{nf}{NF}{nfunk:keinetechniker}
		\item [Name:] Technische Fähigkeiten
		\item [Beschreibung:] Besondere technische Fähigkeiten sind von der Benutzern nicht zu erwarten.
		\item [Motivation:] Auch die Menschen, die kaum etwas von Bedienung bzw. Programmierung von Rechnern verstehen, sollen fähig sein, das System zu verwenden.
		\item [Erfüllungskriterium:]Jeder Benutzer kann die verschiedenen Funktionen intuitiv bedienen und spielen.
	\end{lhp}
\end{description}

\section{Leistungsanforderungen}

\begin{description}[leftmargin=5em, style=sameline]	
	\begin{lhp}{nf}{NF}{nfunk:antwortzeit}
		\item [Name:] Antwortzeit
		\item [Beschreibung:] Maximale Antwortzeit für alle Systemprozesse.
		\item [Motivation:] Das System muss immer brauchbar sein.
		\item [Erfüllungskriterium:] Das System antwortet auf Benutzerhandlungen nie später als in 10 Sekunden.
	\end{lhp}
\end{description}

\section{Anforderungen an Einsatzkontext}

\subsection{Anforderungen an physische Umgebung}

\begin{description}[leftmargin=5em, style=sameline]	
	\begin{lhp}{nf}{NF}{nfunk:beispiel1}
		\item [Name:] Lauffähigkeit an SCI-Rechnern
		\item [Beschreibung:] Das Produkt muss auf einem eigenem Geräte lauffähig sein, welches zur Präsentation am Ende des SEPs genutzt werden muss. Falls keine eigenen Rechner vorhanden, stehen auch die SCI-Terminals zur verfügung.
		\item [Motivation:] Optimierung von Betreuung und Abnahme des SEP/MP
		\item [Erfüllungskriterium:] Das fertige Produkt kann ohne Probleme an den vorhandenen Rechnern vorgeführt werden.
	\end{lhp}
\end{description}


%\subsection{Anforderungen an benachbarte Systeme}
%(sehe Systemkontext)
%
%\begin{description}[leftmargin=5em, style=sameline]	
%	\begin{lhp}{nf}{NF}{nfunk:beispiel2}
%		\item [Name:] Beispiel
%		\item [Beschreibung:] 
%		\item [Motivation:] 
%		\item [Erfüllungskriterium:] 
%	\end{lhp}
%\end{description}

\subsection{Absatz- sowie Installationsbezogene Anforderungen}

\begin{description}[leftmargin=5em, style=sameline]	
	\begin{lhp}{nf}{NF}{nfunk:beispiel3}
		\item [Name:] Installationsanleitung	
		\item [Beschreibung:] Falls die Installation nicht lediglich das Öffnen einer Datei voraussetzt, muss der genaue Installations- und Startvorgang schriftlich für Benutzer zur Verfügung gestellt.
		\item [Motivation:] Spezifikation
		\item [Erfüllungskriterium:] Eventuell auftretende Probleme bei der Installation können von jedem Nutzer anhand der Anleitung umgangen werden.
	\end{lhp}
\end{description}

\subsection{Anforderungen an Versionierung}

\begin{description}[leftmargin=5em, style=sameline]	
	\begin{lhp}{nf}{NF}{nfunk:beispiel4}
		\item [Name:] Keine weitere Versionen
		\item [Beschreibung:] Nach Version 1.0 keine weitere Entwicklung ist vorgesehen.
		\item [Motivation:] Das ist nur SEP/MP, keine Geschäftsprojekt, siehe \ref{fa:fortentwicklung}
		\item [Erfüllungskriterium:] Es werden keine Updates hinzugefügt.
	\end{lhp}
\end{description}

\section{Anforderungen an Wartung und Unterstützung}

\subsection{Wartungsanforderungen}

\begin{description}[leftmargin=5em, style=sameline]	
	\begin{lhp}{nf}{NF}{nfunk:beispiel4}
		\item [Name:] Wartung
		\item [Beschreibung:] Es ist keine Wartung von Seiten der Programmierer oder Dritter vorgesehen. 
		\item [Motivation:] Es sind weder Updates noch eine weitere Verwendung des Programms nach der Abnahme vorgesehen.
		\item [Erfüllungskriterium:] Das Programm wird nach der Abschlusspräsentation nicht weiterentwickelt.
	\end{lhp}
\end{description}

\begin{description}[leftmargin=5em, style=sameline]	
	\begin{lhp}{nf}{NF}{nfunk:doku}
		\item [Name:] Dokumentation
		\item [Beschreibung:] Der Quellcode muss ausführlich dokumentiert werden.
		\item [Motivation:] HiWis und der betreuende Professor sollen Entwicklungsschritte schneller nachvollziehen können.
		\item [Erfüllungskriterium:] JavaDoc 
	\end{lhp}
\end{description}

\begin{description}[leftmargin=5em, style=sameline]	
	\begin{lhp}{nf}{NF}{nfunk:doku}
		\item [Name:] Testen
		\item [Beschreibung:] Der Quellcode außer GUI muss gut getestet werden.
		\item [Motivation:] Fehler sollen frühzeitig erkannt und ausgebessert werden.
		\item [Erfüllungskriterium:] Von Unit-Tests muss mindestens 70\% des Quellcodes bedeckt werden. GUI-Klassen sind aus der Anforderung ausgenommen.
	\end{lhp}
\end{description}

\subsection{Anforderungen an technische und fachliche Unterstützung}

\begin{description}[leftmargin=5em, style=sameline]	
	\begin{lhp}{nf}{NF}{nfunk:beispiel5}
		\item [Name:] Beispiel
		\item [Beschreibung:] Keine technische und fachliche Unterstützung des Systems ist geplannt.
		\item [Motivation:] Siehe \ref{fa:fortentwicklung}.
		\item [Erfüllungskriterium:] Nicht anwendbar.
	\end{lhp}
\end{description}

\subsection{Anforderungen an technische Kompatibilität}

\begin{description}[leftmargin=5em, style=sameline]	
	\begin{lhp}{nf}{NF}{nfunk:beispiel6}
		\item [Name:] Technische Kompatibilität
		\item [Beschreibung:] Das Programm interagiert korrekt mit aller vorhergesehenen Soft- und Hardware.
		\item [Motivation:] Das Programm wird nicht von verschiedenen Kompatibilitätsfaktoren eingeschränkt.
		\item [Erfüllungskriterium:] Das Programm kann in den vorgegebenen Entwicklungsumgebungen eingesetzt werden.
	\end{lhp}
\end{description}

\section{Sicherheitsanforderungen}

\subsection{Zugang}

\begin{description}[leftmargin=5em, style=sameline]	
	\begin{lhp}{nf}{NF}{nfunk:beispiel7}
		\item [Name:] Zugang
		\item [Beschreibung:] Es wird nur Spielern Zugang gewährt, die ein registriertes Konto besitzen und sich darüber anmelden. 
		\item [Motivation:] Spieler und ihre Aktionen sollen auf ein Konto zurückverfolgbar sein.
		\item [Erfüllungskriterium:] Niemand kann die Software ohne gültigen Account nutzen.
	\end{lhp}
\end{description}

\subsection{Integrität}

\begin{description}[leftmargin=5em, style=sameline]	
	\begin{lhp}{nf}{NF}{nfunk:beispiel8}
		\item [Name:] Modifikation des Spiels
		\item [Beschreibung:] Spiele können nicht von anderen Spielern verändert werden.
		\item [Motivation:] Schummeln und Ärgern anderer Mitspieler ist unerwünscht.
		\item [Erfüllungskriterium:] Es wird bei der Implementierung auf gängige Sicherheitsverfahren geachtet. Die Spieler haben keine Möglichkeit, über die Programmoberfläche in den Quellcode des Spiels einzugreifen. 
	\end{lhp}
\end{description}

\subsection{Datenschutz/Privatsphäre}

\begin{description}[leftmargin=5em, style=sameline]	
	\begin{lhp}{nf}{NF}{nfunk:beispiel9}
		\item [Name:] Datenschutz/Privatsphäre
		\item [Beschreibung:] Private Spielerdaten sollen nicht von Dritten eingesehen oder für diese nutzbar gemacht werden.
		\item [Motivation:] Der Nutzer hat ein Recht darauf, dass die genutzte Software Datenschutz gewährleistet.
		\item [Erfüllungskriterium:] Es wird kein Rechtsbruch durch die Nutzung unserer Software auftreten.
	\end{lhp}
\end{description}


\subsection{Virenschutz}

\begin{description}[leftmargin=5em, style=sameline]	
	\begin{lhp}{nf}{NF}{nfunk:beispiel10}
		\item [Name:] Virenschutz
		\item [Beschreibung:] Es ist kein explizierter Virenschutz vorgesehen.
		\item [Motivation:] Die Software läuft lediglich über ein lokales Netzwerk, wodurch keine Malware von außen Einfluss nehmen kann.
		\item [Erfüllungskriterium:] Weder innerhalb unserer Software noch in anderen Programmen lassen sich Viren finden, die durch Sicherheitslücken in unserer Implementierung entstanden sind.
	\end{lhp}
\end{description}

\section{Prüfungsbezogene Anforderungen}

Anforderungen, die sich auf die Prüfung/Audit vom System von SEP/MP-Tutoren oder von weiteren Instanzen beziehen.


\begin{description}[leftmargin=5em, style=sameline]	
	\begin{lhp}{nf}{NF}{nfunk:beispiel10}
		\item [Name:] Formate der Systemdokumentation
		\item [Beschreibung:] Systemdokumantation muss in 2 Formen geführt werden (wenn anwendbar): Die Ausgangsdateien (\LaTeX, Dateien von Diagrammensoftware, von Grafiksoftware usw.) und PDFs.
		\item [Motivation:] Optimierung der SEP/MP-Betreuung.
		\item [Erfüllungskriterium:] Siehe Beschreibung.
	\end{lhp}
\end{description}

\section{Kulturelle und politische Anforderungen}


\begin{description}[leftmargin=5em, style=sameline]	
	\begin{lhp}{nf}{NF}{nfunk:beispiel11}
		\item [Name:] Systemsprache
		\item [Beschreibung:] Die Interfacesprache ist Deutsch.
		\item [Motivation:] Synchronisation des Verständnisses von Teammitgliedern mit unterschiedlichen kulturellen Hintergrunden.
		\item [Erfüllungskriterium:] Es tauchen keine weiteren Sprachen (Anglizismen ausgenommen) in der Software auf.
	\end{lhp}
\end{description}

\section{Rechtliche und standartsbezogene Anforderungen}


\begin{description}[leftmargin=5em, style=sameline]	
	\begin{lhp}{nf}{NF}{nfunk:beispiel12}
		\item [Name:] Nicht rechtliche Anforderungen
		\item [Beschreibung:] Es gibt keine relevanten rechtliche Anforderungen bekannt.
		\item [Motivation:] Siehe \ref{fa:fortentwicklung}.
		\item [Erfüllungskriterium:] Nicht anwendbar.
	\end{lhp}
\end{description}
