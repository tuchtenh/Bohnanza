\chapter{Projektbeschränkungen}

\section{Beschränkungen}

\newcounter{lb}\setcounter{lb}{10}

\begin{description}[leftmargin=5em, style=sameline]
	
	\begin{lhp}{lb}{LB}{beschr:lehrbots}
		\item [Name:] Selbstlehrende Bots
		\item [Beschreibung:] Keine Selbstlehrfunktion von Bots wird implementiert.
		\item [Motivation:] Die Funktionalität ist zu aufwändig zu implementieren und passt deshalb nicht in das Zeitbudget.
		\item [Erfüllungskriterium:] Intelligenzalgorithmus von Bots ist so vorprogrammiert, dass sie Entscheidungen nur anhand des vorprogrammierten Wissens sowie des aktuellen Spielstands treffen, ohne dabei frühere Spiele zu berücksichtigen.
	\end{lhp}
	
	\begin{lhp}{lb}{LB}{beschr:anwendungsbereich}
		\item [Name:] Anwendungsbereich
		\item [Beschreibung:] Das System ist ausschließlich für den privaten Bereich ausgelegt.
		\item [Motivation:] Im Laufe des Projekts muss die Software lediglich privat getestet und vorgeführt werden.
		\item [Erfüllungskriterium:] Das Programm kann fehlerfrei vorgeführt werden.
	\end{lhp}
	
		
	\begin{lhp}{lb}{LB}{beschr:implsprache}
		\item [Name:] Implementierungssprache
		\item [Beschreibung:] Für die Implementierung ist ausschließlich Java 8 oder höher zu verwenden.
		\item [Motivation:] Das optimiert die Betreuung von SEP/MP und koordiniert die Mitarbeit.
		\item [Erfüllungskriterium:] Die Software wurde einheitlich mittels Java 8 implementiert.
	\end{lhp}
	
	\begin{lhp}{lb}{LB}{beschr:gui}
		\item [Name:] GUI-Framework
		\item [Beschreibung:] Die GUI ist mit JavaFX zu realisieren.
		\item [Motivation:] Das optimiert die Betreuung von SEP/MP und koordiniert die Mitarbeit.
		\item [Erfüllungskriterium:] Fenster und ihre Funktionen sind für die Nutzer intuitiv gestaltet und somit von von Spieler einfach zu bedienen.
	\end{lhp}
	
	\begin{lhp}{lb}{LB}{beschr:gitlab}
		\item [Name:] Gitlab
		\item [Beschreibung:] Für die Entwicklung ist das vorgegebene GitLab-Repository zu verwenden.
		\item [Motivation:] Das optimiert die Betreuung von SEP/MP und koordiniert die Mitarbeit.
		\item [Erfüllungskriterium:] Die Struktur und Interaktion zwischen den Softwareabschnitten wurde bereits frühzeitig in der Architektur festgehalten.
	\end{lhp}
	
	
\end{description}

\section{Glossar}

\begin{center}
		\rowcolors{2}{Gray!15}{White}
		\begin{longtable}{p{0.25\textwidth} p{0.25\textwidth} p{0.4\textwidth}}
			\textbf{Deutsch} & \textbf{Englisch} & \textbf{Bedeutung} \\
			\hline \hline \endhead                   
			Bot & bot & Spieler, dessen Spielaktionen vom Computer entschieden und durchgeführt werden\\
			Kekse & Cookies & Offiziell keine gültige Maßnahme zur Bestechung der HiWis\\          
 			Lobby & lobby & Virtueller Raum zum Betreten eines Spielraums\\	
			Spiel (Regelwerk) & game & Bohnanza \\
			Spieler & player & Teilnehmer am Spielgeschehen\\
			Spielraum & game room & Virtueller Raum, in dem ein Spiel stattfindet\\
			Zug & turn & Zustand in dem ein Spieler eine Spielaktion ausführen muss\\
			Spin-off & spinoff & Eigenständiges Spiel mit grundlegender Abwandlung des originalen Spielprinzips \\
			Bohne & bean & Bezeichnung für die in verschiedene Kategorien eingeteilten Spielkarten\\
			Taler & thaler &  Eine Währung im Spiel\\
			Bohnenfeld & field & Ablage, auf der Bohnen im Spielverlauf "angebaut" werden\\
			Hand & hand & Karten, die der Spieler in der Hand hält\\
			Bohnenhandel & bean trading & Möglichkeit der Spieler innerhalb bestimmter Spielphasen Karten miteinander zu 					tauschen\\
			Bohnenschutzregel & bean protection policy & Eine Regel, die das Abbauen von Bohnen vom Feld einschränkt		
		\end{longtable}
\end{center}

\section{Relevante Fakten und Annahmen}

Wichtige gekannte Fakten und getroffene Annahmen, die sich auf das Projekt direkt oder indirekt beziehen und dadruch auf die zukünftige Implementierungsentscheidungen Effekt haben können.

\newcounter{fa}\setcounter{fa}{10}

\begin{description}[leftmargin=5em, style=sameline]
	
	\begin{lhp}{fa}{FA}{fa:fortentwicklung}
		\item [Name:] Keine Fortentwicklung des App nach SEP/MP.
		\item [Beschreibung:] Nach Ende des SEP/MP wird das Projekt nicht weiterentwickelt.
		\item [Motivation:] Das Entwicklungsteam hat keinen Bock darauf.
	\end{lhp}
	
	\begin{lhp}{fa}{FA}{fa:recht}
		\item [Name:] Keine Lizenzen für Spielartefakte.
		\item [Beschreibung:] Weder TU Kaiserslautern noch die Spielwerk + Freizeit GmbH gewahren dem Entwicklungsteam die Rechte für die Spielartefakte.
		\item [Motivation:] Rechtliche Vorsorge.
	\end{lhp}
	
	\begin{lhp}{fa}{FA}{fa:recht-vergangenheit}
		\item [Name:] Keine bekannte Nachteile von Verwendung von Spielartefakten.
		\item [Beschreibung:] Es ist nicht bekannt, dass die SEP/MP-Teilnehmer der letzten Jahre irgendwelche rechtliche Probleme dadurch gehabt haben, dass sie die Speilartefakten von Spielwerk + Freizeit GmbH im Rahmen ihrer SEP/MP eingesetzt haben.
		\item [Motivation:] Rechtliche Vorsorge.
	\end{lhp}
	
	
\end{description}

