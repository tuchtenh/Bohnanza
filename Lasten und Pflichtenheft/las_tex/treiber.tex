\chapter{Projekttreiber}

\section{Projektziel}

Im Rahmen des Software-Entwicklungs-Projekts/Modellierungspraktikums {\the\year} soll ein einfach zu bedienendes Client-Server-System zum Spielen von Bohnanza über ein Netzwerk implementiert werden. Die Benutzeroberfläche soll intuitiv bedienbar sein.

\section{Stakeholders}

\newcounter{sh}\setcounter{sh}{10}

\begin{description}[leftmargin=5em, style=sameline]
	
	\begin{lhp}{sh}{SH}{sh:Spieler}
		\item [Name:] Spieler
		\item [Beschreibung:] Menschliche Spieler.
		\item [Ziele/Aufgaben:] Das Spiel zu spielen.
	\end{lhp}
	
	\begin{lhp}{sh}{SH}{bsh:Spieler}
		\item [Name:] Eltern
		\item [Beschreibung:] Eltern minderjähriger Spieler.
		\item [Ziele/Aufgaben:] Um die Spieler zu kümmern, indem Eltern Spielzeit begrenzen wollen und zugriff auf sensible Inhalte begrenzen.
	\end{lhp}
	
	\begin{lhp}{sh}{SH}{bsh:gesetzgeber}
		\item [Name:] Gesetzgeber
		\item [Beschreibung:] Das Amt für Jugend und Familie.
		\item [Ziele/Aufgaben:] Die Rechte der Spieler zu schützen und zu gewähren, indem er Gesetze erstellt.
	\end{lhp}
	
	\begin{lhp}{sh}{SH}{bsh:investor}
		\item [Name:] Investoren (nur für Beispielzwecken)
		\item [Beschreibung:] Parteien, die das Finanzmittel für die Entwicklung des Systems bereitstellen.
		\item [Ziele/Aufgaben:] Gewinn zu ermitteln, indem das System an Endverbraucher verkauft wird.
	\end{lhp}
	
	\begin{lhp}{sh}{SH}{bsh:betreuer}
		\item [Name:] Betreuer
		\item [Beschreibung:] HiWis, die SEP/MP Projektgruppen betreuen.
		\item [Ziele/Aufgaben:] Das Entwicklungsprozess zu betreuen, zu überwachen und teilweise zu steuern als auch die Arbeit der Projektgruppen abzunehmen sowie den Studenten im Prozess Hilfe zur Verfügung zu stellen. 
	\end{lhp}
	
	\begin{lhp}{sh}{SH}{bsh:prof}
		\item [Name:] Prof. Dr. Achim Ebert
		\item [Beschreibung:] Betreuer des Gesamtprojekts 
		\item [Ziele/Aufgaben:] Als zentraler Ansprechpartner betreut er in Zusammenarbeit mit den HiWis das Projekt und bewertet es schlussendlich.
	\end{lhp}
		
\end{description}

\section{Aktuelle Lage}

Aktuell wird Bohnanza als Familien-Kartenspiel verwendet und besitzt (für seine Originalvariante) keine technisch realisierte Anwendung. Es sind diverse Erweiterungen, wie auch Spin-Offs des Titels erschienen, auf die in diesem Projekt aus zeitlichen Gründen nicht weiter eingegangen wird. Für uns ist es wichtig, eine solide Alternative zum klassischen Kartenspiel zu entwerfen, die dem Original durch angemessene Musik, Grafik und Interaktion neue Farbe verleiht. \\
Im weiteren Verlauf sind weder Erweiterungen noch Online-Varianten geplant.