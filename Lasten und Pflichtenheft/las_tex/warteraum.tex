\chapter{Warteraum}

Hier werden Anforderungen spezifiziert die den sogenannten ``Warteraum'' darstellen. Hier gehören alle Anforderungen, die ``Wünschkriterien'' sind, das heißt, sie sind zwar erwünscht, aber werden nur dann in aktuelle Anforderungen übernommen, wenn dafür genügende Zeitbudget vorhanden ist und werden am wahrscheinlichsten in der Zukunft und nicht jetzt implementiert (oder in den kommenden Sprints beim SCRUM-Prozessmodell).

\newcounter{wr}\setcounter{wr}{10}

\begin{description}[leftmargin=5em, style=sameline]	
	\begin{lhp}{wr}{WR}{nfunk:sarch1}
		\item [Name:] Hintergrundmusik
		\item [Beschreibung:] Für die Spieler soll ein Auswahl zur Verfügung stehen, was für Hintergrundmusik sie beim Spielen in der Anwendung hören.
		\item [Motivation:] Höhere Zufriedenheit der Benutzer
		\item [Erfüllungskriterium:] Spieler können an jedem Zeitpunkt außer Vorraum aus einem Auswahl eine Musik auswählen oder die Musik ausschalten.
	\end{lhp}
\end{description}

\begin{description}[leftmargin=5em, style=sameline]	
	\begin{lhp}{wr}{WR}{nfunk:stats}
		\item [Name:] Stats der Spieler
		\item [Beschreibung:] In der Bestenliste soll es möglich sein, eine erweiterte Ansicht mit gewonnenen und verlorenen Spielen des Spielers zu sehen, sowie dessen höchst gewonnenes Spiel.
		\item [Motivation:] Höhere Zufriedenheit und mehr Ehrgeiz der Benutzer
		\item [Erfüllungskriterium:] Spieler können auf die Namen in der Bestenliste klicken.
	\end{lhp}
\end{description}

\begin{description}[leftmargin=5em, style=sameline]	
	\begin{lhp}{wr}{WR}{nfunk:freund}
		\item [Name:] Freundesliste
		\item [Beschreibung:] Es soll möglich sein, eine Freunde hinzuzufügen und zu sehen, ob diese online sind.
		\item [Motivation:] Höhere Zufriedenheit der Benutzer und mehr soziale Interaktion
		\item [Erfüllungskriterium:] Spieler können andere Spieler durch ein einen Button zur Freundesliste hinzufügen oder bestimmte Spielernamen suchen.
	\end{lhp}
\end{description}

\begin{description}[leftmargin=5em, style=sameline]	
	\begin{lhp}{wr}{WR}{nfunk:bots}
		\item [Name:] Bots
		\item [Beschreibung:] Es sollen mehr als zwei Schwierigkeitsstufen für Bots hinzugefügt werden.
		\item [Motivation:] Höhere Zufriedenheit der Benutzer
		\item [Erfüllungskriterium:] Spieler können aus mehreren Schwierigkeitsstufen für Bots wählen.
	\end{lhp}
\end{description}

\begin{description}[leftmargin=5em, style=sameline]	
	\begin{lhp}{wr}{WR}{nfunk:tut}
		\item [Name:] Tutorial
		\item [Beschreibung:] Beim ersten Login werden alle Menüpunkte durch Popups bzw. Tooltips erläutert.
		\item [Motivation:] Höhere Zufriedenheit der Benutzer und leichter Einstieg ins Spiel
		\item [Erfüllungskriterium:] Spieler erhalten beim ersten Login eine schrittweise Einführung in die Menüführung.
	\end{lhp}
\end{description}