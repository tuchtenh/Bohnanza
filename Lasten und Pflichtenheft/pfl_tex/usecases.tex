\section{Beschreibungen der Anwendungsfälle}



\newcounter{uc}\setcounter{uc}{10}

\begin{description}[leftmargin=5em, style=sameline]

	\begin{lhp}{uc}{UC}{uc:regist}
		\item [Name:] Spieler registrieren.
		\item [Ziel:] Spieler kann Bohnanza spielen. 
		\item [Akteure:] Spieler.
		\item [Vorbedingungen:] Bohnanza wurde erfolgreich installiert und gestartet.
		\item [Eingabedaten:] Zugriffsdaten bestehend aus Benutzername, Passwort und Passwort-Wiederholung
		\item [Beschreibung:] \hfill\\ \hfill\\
				1. Der Spieler installiert und öffnet Bohnanza.\\
				2. Der Spieler öffnet den Registrieren-Dialog. \\
				3. Der Spieler gibt seine Daten ein und sendet das Formular ab.\\
				4. Das System prüft die Gleichheit von Passwort und Passwort-Wiederholung.\\
				5. Das System prüft auf bereits vorhandene Spieler des gleichen Namens.\\
				6. Das System legt einen neuen Eintrag in der Datenbank an. \\
				7. Der System bestätigt die erfolgreiche Registrierung und leitet den Spieler in die Lobby weiter.\\
		\item [Ausnahmen:] \hfill
			\begin{itemize} 
				\item[] \textit{Passwort und Passwort-Wiederholung stimmen nicht überein:} Das System zeigt eine Fehlermeldung an und fordert den Benutzer auf, das Passwort erneut einzugeben.
				\item[] \textit{Benutzername bereits vorhanden:} Das System zeigt eine Fehlermeldung an und fordert den Benutzer auf, einen anderen Benutzernamen zu wählen.
			
			\end{itemize}
		\item [Ergebnisse und Outputdaten:] Neuer Eintrag (Spieler) in der Datenbank. Spieler befindet sich in der Lobby.
		\item [Systemfunktionen] \ref{funk:zugriff}.
	\end{lhp}
	
	\begin{lhp}{uc}{UC}{uc:anmeld}
		\item [Name:] Spieler anmelden.
		\item [Ziel:] Spieler meldet sich im System an.
		\item [Akteure:] Spieler.
		\item [Vorbedingungen] Spieler ist im Vorraum.
		\item [Eingabedaten:] Zugriffsdaten.
		\item [Beschreibung:] \hfill\\ \hfill\\
			1. Spieler sendet das Formular ab.\\
			2. Das System prüft die Gültigkeit von Zugangsdaten und bewegt den Spieler in Lobby.\\				
		\item [Ausnahmen:] \hfill
			\begin{itemize} 
				\item[] \textit{Passwort oder Benutzername ist falsch:} Das System zeigt eine Fehlermeldung an, anstatt des Schrittes 2.
				
			\end{itemize}
		\item [Ergebnisse und Outputdaten:] Spieler ist im Lobby und sieht die Bestenliste.	
		\item [Systemfunktionen:] \ref{funk:zugriff}.
	\end{lhp}
	
	\begin{lhp}{uc}{UC}{uc:loeschen}
		\item [Name:] Spieler löschen.
		\item [Ziel:] Spieler entfernt seine Daten aus dem System.
		\item [Akteure:] Spieler.
		\item [Vorbedingungen] Spieler ist im Vorraum.
		\item [Eingabedaten:] Passwort.
		\item [Beschreibung:] \hfill\\ \hfill\\
				1. Spieler sendet das Formular ab.\\
				2. Das System prüft die Richtigkeit vom Passwort und fragt Spieler noch ein mal, ob er sich wirklich aus dem System entfernen möchte.\\
				3. Spieler bestätigt seine Intention.\\
				4. Das System entfern alle Daten des Spielers aus der Datenbank und bewegt Spieler in den Vorraum.\\
		\item [Ausnahmen:] \hfill
			\begin{itemize} 
				\item[] \textit{Passwort ist falsch:} Das System zeigt eine Fehlermeldung an, anstatt des Schrittes 2.
				\item[] \textit{Keine Löschung erwünscht:} Anstatt des Schrittes 4, schließt das System das Dialog.
				
			\end{itemize}
		\item [Ergebnisse und Outputdaten:] Spieler ist im Vorraum, Spielerkonto wurde gelöscht.	
		\item [Systemfunktionen:] \ref{funk:zugriff}.
	\end{lhp}

	\begin{lhp}{uc}{UC}{uc:abmelden}
		\item [Name:] Spieler abmelden
		\item [Ziel:] Spieler meldet sich vom System ab.
		\item [Akteure:] Spieler.
		\item [Vorbedingungen] Spieler ist im Vorraum.
		\item [Eingabedaten:] Betätigung des Buttons. 
		\item [Beschreibung:] \hfill\\ \hfill\\
			1. Spieler betätigt den Abmeldebutton.\\
			2. Das System fragt, ob die Abmeldung wirklich erfolgen soll.\\
			3. Spieler bestätigt seine Intention.\\
			4. Das System meldet den Spieler vom Spiel ab.\\
		\item [Ausnahmen:] \hfill
			\begin{itemize} 
				\item[] \textit{Keine} 
		
			\end{itemize}
		\item [Ergebnisse und Outputdaten:] Spieler ist vom System abgemeldet und Programm wird geschlossen. 	
		\item [Systemfunktionen:] \ref{funk:zugriff}.
	\end{lhp}

	\begin{lhp}{uc}{UC}{uc:spielraumerst}
		\item [Name:] Neuen Spielraum erstellen.
		\item [Ziel:] Der Spieler erstellt ein neuen Spielraum in der Lobby
		\item [Akteure:] Spieler.
		\item [Vorbedingungen] Spieler ist in der Lobby.
		\item [Eingabedaten:] Name des Spielraumes, Anzahl der maximalen Spieler (3-5).
		\item [Beschreibung:] \hfill\\ \hfill\\
			1. Spieler betätigt "{}Spiel erstellen{}"{}-Button.\\
			2. Das System fordert den Benutzer auf einen Spielnamen und die Anzahl der maximalen Spieler einzugeben.\\
			3. Spieler bestätigt seine Intention.\\
			4. Das System prüft, ob dieser Spielname bereits vorhanden ist.	\\		
			5. Das System erstellt einen neuen Spielraum in der Lobby. \\
		\item [Ausnahmen:] \hfill
			\begin{itemize} 
				\item[] \textit{Spielname bereits vorhanden:} Das System zeigt eine Fehlermeldung an und fordert den Spieler auf, einen neuen Spielnamen einzugeben.
				
			\end{itemize}
		\item [Ergebnisse und Outputdaten:] Neues Spiel in der Lobby ersichtlich. Spieler ist dem Spiel als "{}Spielersteller{}"{} zugeordnet.  	
		\item [Systemfunktionen:] \ref{funk:spielraum}.
	\end{lhp}

	\begin{lhp}{uc}{UC}{uc:spielraumaend}
		\item [Name:] Vorhandenen Spielraum ändern.
		\item [Ziel:] Der Spielraumersteller ändert die Einstellungen des Spielraumes.
		\item [Akteure:] Spieler (Spielersteller).
		\item [Vorbedingungen] Spieler ist im Spielraum.
		\item [Eingabedaten:] Name des Spielraumes, Anzahl der maximalen Spieler (3-5), Anzahl der Bots (1-4).
		\item [Beschreibung:] \hfill\\ \hfill\\
			1. Spieler betätigt "{}Ändern{}"{}-Button.\\
			2. Das System fordert den Benutzer auf den Spielnamen, die Anzahl der maximalen Spieler und die Anzahl der Bots zu ändern. \\
			3. Spieler bestätigt seine Intention.\\
			4. Das System prüft, ob dieser Spielname bereits vorhanden ist.\\	
			5. Das System prüft, ob die Anzahl der zulässigen maximalen Spieler durch die Änderungen und durch das Hinzufügen der Bots überschritten wurde.\\	
			5. Das System ändert die Daten des Spielraumes.\\
		\item [Ausnahmen:] \hfill
			\begin{itemize} 
				\item[] \textit{Spielname bereits vorhanden:} Das System zeigt eine Fehlermeldung an und fordert den Spieler auf, einen anderen Spielnamen einzugeben.
				\item[] \textit{Anzahl der maximalen Spieler überschritten:} Das System zeigt eine Fehlermeldung an und fordert den Spieler auf, Bots zu entfernen oder die Anzahl der maximalen Spieler zu erhöhen.
				
			\end{itemize}
		\item [Ergebnisse und Outputdaten:] Daten des Spielraumes aktualisiert.   	
		\item [Systemfunktionen:] \ref{funk:spielraum}.
	\end{lhp}

	\begin{lhp}{uc}{UC}{uc:spielraumloe}
		\item [Name:] Spielraum löschen.
		\item [Ziel:] Der Spielraumersteller löscht den Spielraum.
		\item [Akteure:] Spieler (Spielersteller).
		\item [Vorbedingungen] Spieler ist im Spielraum.
		\item [Eingabedaten:] Bestätigungs-Button.
		\item [Beschreibung:] \hfill\\ \hfill\\
			1. Spieler betätigt "{}Löschen{}"{}-Button.\\
			2. Das System fragt nach, ob der Spielraum wirklich gelöscht werden soll.\\
			3. Spieler bestätigt seine Intention.\\
			4. Das System löscht den Spielraum.	\\	
			5. Das System leitet den Spielersteller und weitere Spieler in die Lobby zurück.\\	
			5. Das System meldet "{}Spielraum gelöscht{}"{}.\\
		\item [Ausnahmen:] \hfill
			\begin{itemize} 
				\item[] \textit{keine} 
			
			\end{itemize}
		\item [Ergebnisse und Outputdaten:] Spielraum gelöscht. 	
		\item [Systemfunktionen:] \ref{funk:spielraum}.
	\end{lhp}

	\begin{lhp}{uc}{UC}{uc:spielraumbeit}
		\item [Name:] Spielraum beitreten.
		\item [Ziel:] Der Spieler tritt einem Spielraum bei.
		\item [Akteure:] Spieler.
		\item [Vorbedingungen] Spieler ist in der Lobby.
		\item [Eingabedaten:] Bestätigungs-Button.
		\item [Beschreibung:] \hfill\\ \hfill\\
			1. Spieler betätigt "{}Beitreten{}"{}-Button.\\
			2. Das System prüft, ob die maximale Anzahl der Spieler überschritten wird.\\
			3. Das System fügt den Spieler dem Spielraum hinzu.\\	
			4. Das System meldet den anderen Teilnehmern "{}Neuer Spieler beigetreten{}"{}.\\
		\item [Ausnahmen:] \hfill
			\begin{itemize} 
				\item[] \textit{Spielraum voll:} Das System meldet dem Spieler:{} {}Spielraum bereits voll, bitte ein anderes Spiel auswählen{}"{}.
				
			\end{itemize}
		\item [Ergebnisse und Outputdaten:] Spielraum gelöscht. 	
		\item [Systemfunktionen:] \ref{funk:spielraum}.
	\end{lhp}

	\begin{lhp}{uc}{UC}{uc:spielraumverl}
		\item [Name:] Spielraum verlassen.
		\item [Ziel:] Der Spieler verlässt den Spielraum
		\item [Akteure:] Spieler.
		\item [Vorbedingungen] Spieler ist im Spielraum.
		\item [Eingabedaten:] Bestätigungs-Button.
		\item [Beschreibung:] \hfill\\ \hfill\\
			1. Der Spieler betätigt "{}Verlassen{}"{}-Button.\\
			2. Das System fordert Bestätigung.\\
			3. Der Spieler bestätigt seine Intention.\\
			4. Das System entfernt den Spieler aus dem Spielraum und leitet ihn in die Lobby zurück.\\	
			4. Das System meldet den anderen Teilnehmern "{}Spieler hat Spielraum verlassen{}"{}.	
		\item [Ausnahmen:] \hfill
			\begin{itemize} 
				\item[] \textit{Spielerersteller verlässt den Spielraum:} Das System meldet dem Spieler: "{}Bitte Spielraum zuerst löschen{}"{}.
					
			\end{itemize}
		\item [Ergebnisse und Outputdaten:] Spielraum verlassen. Spieler in Lobby.	
		\item [Systemfunktionen:] \ref{funk:spielraum}.
	\end{lhp}	
	

	\begin{lhp}{uc}{UC}{uc:spielstart}
		\item [Name:] Spiel starten.
		\item [Ziel:] Der Spielraumersteller startet das Spiel.
		\item [Akteure:] Spieler (Spielersteller).
		\item [Vorbedingungen] Spieler ist im Spielraum. Die Anzahl der Spieler inklusive Bots im Spielraum beträgt mindestens 3.
		\item [Eingabedaten:] Bestätigungs-Button.
		\item [Beschreibung:] \hfill\\ \hfill\\
			1. Spieler betätigt "{}Starten{}"{}-Button.\\
			2. Das System fragt nach, ob das Spiel gestartet werden soll.\\
			3. Spieler bestätigt seine Intention.\\
			4. Das System entfernt das Spiel aus dem Spielraum.\\
			5. Das System startet ein Spiel.\\	
			6. Das System bestimmt zufällig einen Startspieler.\\	
			7. Das System vergibt jedem Spieler fünf Handkarten.\\
			8. Das System zeigt eine Meldung {}"Spiel beginnt{}"{}.\\
			9. Das System fordert den Startspieler dazu auf, den Zug zu beginnen.\\
		\item [Ausnahmen:] \hfill
			\begin{itemize} 
					\item[] \textit{Nicht genügend Spieler vorhanden:} Das System zeigt eine Meldung: "{}Nicht genügend Spieler zum Starten des Spiels vorhanden{}"{}.
					
			\end{itemize}
		\item [Ergebnisse und Outputdaten:] Spiel gestartet. Die Spieler befinden sich in der Spieloberfläche.	
		\item [Systemfunktionen:] \ref{funk:spielraum}.
	\end{lhp}

	\begin{lhp}{uc}{UC}{uc:spielverl}
		\item [Name:] Spiel verlassen.
		\item [Ziel:] Der Spieler verlässt das Spiel.
		\item [Akteure:] Spieler.
		\item [Vorbedingungen] Spieler ist im Spiel.
		\item [Eingabedaten:] Bestätigungs-Button.
		\item [Beschreibung:] \hfill\\ \hfill\\
			1. Der Spieler betätigt "{}Verlassen{}"{}-Button.\\
			2. Das System fordert Bestätigung.\\
			3. Der Spieler bestätigt seine Intention.\\
			4. Das System entfernt den Spieler aus dem Spiel und leitet ihn in die Lobby zurück.\\	
			5. Das System meldet den anderen Teilnehmern "Spieler hat das Spiel verlassen".\\
			6. Das System prüft, ob sich weniger als drei Spieler im Spielraum befinden und fügt ggf. einen Bot hinzu.\\
			7. Das System prüft, ob der Startspieler das Spiel verlassen hat und bestimmt einen Startspieler links vom ehemaligen Startspieler.\\
		\item [Ausnahmen:] \hfill
			\begin{itemize} 
				\item[] \textit{Spielerersteller verlässt den Spielraum:} Das System meldet dem Spieler: "{}Spiel wird geschlossen, wenn Sie das Spiel verlassen - denken Sie über Ihre Entscheidung nach{}"{}.
				
			\end{itemize}
		\item [Ergebnisse und Outputdaten:] Spiel verlassen. Spieler befindet sich in Lobby.	
		\item [Systemfunktionen:] \ref{funk:spielverw}.
	\end{lhp}

	\begin{lhp}{uc}{UC}{uc:phase1}
		\item [Name:] Phase 1.
		\item [Ziel:] Phase 1 durchlaufen.
		\item [Akteure:] Spieler.
		\item [Vorbedingungen] Spiel frisch gestartet oder Phase 4 beendet, Handkarten vorhanden.
		\item [Eingabedaten:] Bestätigungs-Button, Handkarten in richtiger Reihenfolge.
		\item [Beschreibung:] \hfill\\ \hfill\\
			1. System prüft, ob menschlicher Spieler oder Bot an der Reihe ist und ruft ggf. \ref{uc:bot} auf.\\
			2. Spieler baut Handkarten an.\\
			3. Der Spieler beendet die Phase.\\
			4. Das System leitet Phase 2 ein.\\
		\item [Ausnahmen:] \hfill
			\begin{itemize} 
				\item[] \textit{Kein freies Anbaufeld vorhanden:} Das System zeigt eine Meldung: {}"{}Kein freies Anbaufeld vorhanden, Felder abernten (\ref{uc:abernten}){}"
				
			\end{itemize}
		\item [Ergebnisse und Outputdaten:] Handkarten angebaut. Phase 1 beendet.	
		\item [Systemfunktionen:] \ref{funk:spielverw}.
	\end{lhp}

	\begin{lhp}{uc}{UC}{uc:abernten}
		\item [Name:] Bohnenkarten abernten.
		\item [Ziel:] Der Spieler baut Bohnenkarten ab.
		\item [Akteure:] Spieler.
		\item [Vorbedingungen] keine
		\item [Eingabedaten:] Bestätigungs-Button.
		\item [Beschreibung:] \hfill\\ \hfill\\
			1. Spieler betätigt "{}Ernten{}"{}-Button.\\
			2. Das System fragt nach, ob die Bohnen geerntet werden sollen.\\
			3. Spieler bestätigt seine Intention.\\
			4. Das System prüft "{}Bohnenschutzregel{}"{}.\\
			4. Das System entfernt die Karten vom Ablagefeld\\
			5. Das System legt Karten in der Anzahl der erhaltenen Taler auf das Talerfeld.\\	
			6. Das System legt die restlichen Karten auf den Ablagestapel.\\
		\item [Ausnahmen:] \hfill
			\begin{itemize} 
				\item[] \textit{Bohnenschutzregel verletzt:} Das System meldet {}"Bohnenschutzregel verletzt, Aktion nicht möglich{}"{}.
				
			\end{itemize}
		\item [Ergebnisse und Outputdaten:] Bohnenfeld leer, Taler des Spielers ggf. erhöht.	
		\item [Systemfunktionen:] \ref{funk:spielraum}.
	\end{lhp}

	\begin{lhp}{uc}{UC}{uc:phase2}
		\item [Name:] Phase 2.
		\item [Ziel:] Phase 2 durchlaufen.
		\item [Akteure:] Spieler.
		\item [Vorbedingungen] Phase 1 beendet.
		\item [Eingabedaten:] Bestätigungs-Button
		\item [Beschreibung:] \hfill\\ \hfill\\
			1. Das System legt zwei Karten vom Nachziehstapel daneben und prüft \ref{uc:spielverwaltung}.\\
			2. Das System fordert den Spieler auf die Karten zu nehmen oder zu handeln (\ref{uc:handeln})\\
			3. Das System wartet ggf. \ref{uc:handeln} ab.\\
			4. Das System legt die übrigen Karten neben dem Nachziehstapel quer neben das Anbaufeld.\\
			5. Der aktive Spieler beendet die Phase.\\
			6. Das System leitet Phase 3 ein.\\
		\item [Ausnahmen:] \hfill
			\begin{itemize} 
				\item[] \textit{keine} 
				
			\end{itemize}
		\item [Ergebnisse und Outputdaten:] Neue Karten liegen neben dem Bohnenfeld. Phase 2 beendet.
		\item [Systemfunktionen:] \ref{funk:spielverw}.
	\end{lhp}

	\begin{lhp}{uc}{UC}{uc:handeln}
		\item [Name:] Handel.
		\item [Ziel:] Handel abschließen.
		\item [Akteure:] Spieler.
		\item [Vorbedingungen] Phase 2 läuft.
		\item [Eingabedaten:] Bestätigungs-Button, Bohnenkarten
		\item [Beschreibung:] \hfill\\ \hfill\\
			1. Der aktive Spieler wählt einen anderen Spieler zum Handeln aus.\\
			2. Die Spieler bestätigen den Handel.\\
			3. Das System tauschen die Karten aus.\\
			4. Der aktive Spieler beendet den Handel.\\
			5. Das System legt die neu erworbenen Karten neben das Bohnenfeld.\\
			6. Der aktive Spieler wiederholt ggf. Schritt 1-4.\\
		7. Der Spieler beendet seine Handelsphase.\\
		\item [Ausnahmen:] \hfill
			\begin{itemize} 
				\item[] \textit{Handel abgebrochen:} Das System "{}Handel abgebrochen{}"{}, wenn einer der Spieler den Handel nicht bestätigt oder das Handelfenster schließt.
				
			\end{itemize}
		\item [Ergebnisse und Outputdaten:] Handelkarten neben dem Bohnenfeld. Handelphase abgeschlossen. 
		\item [Systemfunktionen:] \ref{funk:spielverw}.
	\end{lhp}

	\begin{lhp}{uc}{UC}{uc:phase3}
		\item [Name:] Phase 3.
		\item [Ziel:] Phase 3 durchlaufen.
		\item [Akteure:] Spieler.
		\item [Vorbedingungen] Phase 2 ist abgeschlossen.
		\item [Eingabedaten:] Bestätigungs-Button
		\item [Beschreibung:] \hfill\\ \hfill\\
			1. Das System prüft, ob freie Felder zum Anbauen der Bohnenkarten vorhanden sind.\\
			2. Alle Spieler bauen die Bohnenkarten, erworben durch den Handel (\ref{uc:handeln}) oder vom Nachziehstapel (aktiver Spieler, \ref{uc:phase2}) an.\\
			3. Das System beendet die Phase, sobald alle Spieler ihre Karten angebaut haben.\\
		\item [Ausnahmen:] \hfill
			\begin{itemize} 
				\item[] \textit{Kein freies Anbaufeld:} Das System meldet "{}kein freies Anbaufeld vorhanden{}"{}.
				
			\end{itemize}
		\item [Ergebnisse und Outputdaten:] Erworbene Bohnenkarten aus Phase 2 wurden angebaut.  
		\item [Systemfunktionen:] \ref{funk:spielverw}.
	\end{lhp}

	\begin{lhp}{uc}{UC}{uc:phase4}
		\item [Name:] Phase 4.
		\item [Ziel:] Phase 4 durchlaufen.
		\item [Akteure:] Spieler.
		\item [Vorbedingungen] Phase 3 ist abgeschlossen.
		\item [Eingabedaten:] Bestätigungs-Button
		\item [Beschreibung:] \hfill\\ \hfill\\
			1. Das System prüft \ref{uc:spielverwaltung} und gibt dem aktiven Spieler drei neue Handkarten in entsprechender Reihenfolge.\\
			2. Das System beendet Phase 4.\\
			3. Das System übergibt Spielrunde dem nächsten Spieler und beginnt dort mit Phase 1.\\
		\item [Ausnahmen:] \hfill
			\begin{itemize} 
				\item[] \textit{keine} 
				
			\end{itemize}
		\item [Ergebnisse und Outputdaten:] Drei neue Handkarten erworben. Aktiver Spieler gewechselt.
		\item [Systemfunktionen:] \ref{funk:spielverw}.
	\end{lhp}


	\begin{lhp}{uc}{UC}{uc:bestenliste}
		\item [Name:] Bestenliste.
		\item [Ziel:] Das System zeigt eine Bestenliste an und hält diese aktuell.
		\item [Akteure:] System.
		\item [Vorbedingungen] Es existieren Einträge in der Datenbank.
		\item [Eingabedaten:] Einträge bisher gespielter Spiele.
		\item [Beschreibung:] \hfill\\ \hfill\\
			1. Das System stellt eine Anfrage an die Datenbank.\\
			2. Das System zeigt die von jedem Spieler die Anzahl der gewonnen Spiele an.\\
		\item [Ausnahmen:] \hfill
				\begin{itemize} 
					\item[] \textit{Keine Einträge:} Sind keine Einträge vorhanden, so zeigt das System "{}Keine Einträge vorhanden{}"{}.
					
				\end{itemize}
		\item [Ergebnisse und Outputdaten:] Bestenliste.	
		\item [Systemfunktionen:] \ref{funk:bestenliste}.
	\end{lhp}

	\begin{lhp}{uc}{UC}{uc:chat}
		\item [Name:] Chatverwaltung.
		\item [Ziel:] Das System zeigt im Spielraum, in der Lobby und im Spiel ein Chatfenster.
		\item [Akteure:] System, Spieler.
		\item [Vorbedingungen] Es existiert ein Spiel oder ein Spielraum.
		\item [Eingabedaten:] Chateingaben der Nutzer.
		\item [Beschreibung:] \hfill\\ \hfill\\
			1. Das System zeigt einen Chat an.\\
			2. Spieler können Chatnachrichten absenden.\\
			3. Das System zeigt die Chatnachricht zusammen mit dem Spielernamen an.\\
		\item [Ausnahmen:] \hfill
				\begin{itemize} 
					\item[] \textit{Keine Einträge:} Ist das Chatfenster leer, zeigt das System die Meldung "{}Keine Chatnachrichten vorhanden{}"{}.
					
				\end{itemize}
		\item [Ergebnisse und Outputdaten:] Chat mit aktuellen Chatnachrichten.	
		\item [Systemfunktionen:] \ref{funk:chat}.
	\end{lhp}

	\begin{lhp}{uc}{UC}{uc:bot}
		\item [Name:] Botverwaltung.
		\item [Ziel:] Das System steuert einen Bot.
		\item [Akteure:] System.
		\item [Vorbedingungen] Es existiert ein Spiel oder ein Spielraum.
		\item [Eingabedaten:] Chateingaben der Nutzer.
		\item [Beschreibung:] \hfill\\ \hfill\\
			1. Das System steuert einen Bot nach den Phasen des Spiels:\ref{uc:phase1},\ref{uc:phase2}, \ref{uc:phase3} und \ref{uc:phase4} und nach vorgegebenen Schwierigkeitsstufen \ref{funk:bots}.\\
		\item [Ausnahmen:] \hfill
			\begin{itemize} 
				\item[] \textit{keine} 
				
			\end{itemize}
		\item [Ergebnisse und Outputdaten:] Aktionen des Bots entsprechend der Phasen des Spiels und der Schwierigkeitsstufe des Bots.
		\item [Systemfunktionen:] \ref{funk:bot}.
	\end{lhp}


	\begin{lhp}{uc}{UC}{uc:spielverwaltung}
		\item [Name:] Spielverwaltung.
		\item [Ziel:] Das System verwaltet die Phasen eines Spiels.
		\item [Akteure:] System.
		\item [Vorbedingungen] Das Spiel wurde aus dem Spielraum heraus gestartet.
		\item [Eingabedaten:] Aktion des Spielers.
		\item [Beschreibung:] \hfill\\ \hfill\\
			1. Das System prüft, ob Nachziehstapel leer ist. 
			2. Das Sysem erneuert bei Bedarf den Nachziehstapel mithilfe des Ablegestapels.
			3. Das System prüft, wie oft der Ziehstapel erneuert wurde. Bei dreimaligem Leeren beendet das System nach Ablauf der Phasen 2-3 (Trigger in Phase 2 \ref{uc:phase2}) oder sofort (Trigger in Phase 4 \ref{uc:phase4}) das Spiel.\\
			4. Das System meldet "{}Nach Phase 2 und 3 wird das Spiel beendet oder "Spiel wird beendet{}"{}.\\
			5. Das System ruft usecase "{}Spiel beenden (\ref{uc:spielbeenden}){}" {} auf.\\
		\item [Ausnahmen:] \hfill
				\begin{itemize} 
					\item[] \textit{keine}
					
				\end{itemize}
		\item [Ergebnisse und Outputdaten:] Nächste Phase des Spiels oder Beendigung des Spiels.	
		\item [Systemfunktionen:] \ref{funk:spielraum}.
	\end{lhp}

	\begin{lhp}{uc}{UC}{uc:spielbeenden}
		\item [Name:] Spiel beenden.
		\item [Ziel:] Das System beendet das Spiel.
		\item [Akteure:] System.
		\item [Vorbedingungen] Der Ziehstapel wurde dreimal neu gemischt, die letzten Phasen der Runde sind beendet.
		\item [Eingabedaten:] keine
		\item [Beschreibung:] \hfill\\ \hfill\\
			1. Das System bestimmt den Gewinner des Spiels.\\
			2. Das System meldet den Spielern die Reihenfolge ihrer Platzierung.\\
			3. Das System aktualisiert die Daten des Gewinners in der Datenbank.\\
			4. Das System beendet das Spiel und leitet die Spieler in die Lobby weiter.\\
		\item [Ausnahmen:] \hfill
			\begin{itemize} 
				\item[] \textit{keine}
				
			\end{itemize}
		\item [Ergebnisse und Outputdaten:] Spieler befinden sich in der Lobby. "{}Gewonnene Spiele"{} des Gewinners in der Datenbank aktualisiert.
		\item [Systemfunktionen:] \ref{funk:spielverw}.
	\end{lhp}
		
\end{description}